\section{Introduction}
\label{s:intro}
Software caches, e.g., Memcached~\cite{}, Redis~\cite{}, and Linux page cache~\cite{}, are widely adapted in various fields to improve the service performance.
%
In production environments, software cache systems have a fixed memory space and a specific cache replacement policy for all data sets.
%
In the system, there are various workloads with different access patterns, and some of these patterns change over time. 
%
Many works analyze the characteristics of the traces and design the best policy for the system to improve performance.

Cache replacement policies are designed to handle different access patterns adaptively.
%
Prior work~\cite{} classifies the access patterns into four basic types and designs the policy for the mixture of the basic patterns.
%
To design an adaptive policy, they split the cache into different functional parts to filter the interference pattern from the major pattern and duel for two major patterns.
%
Furthermore, some workloads also exhibit regular access patterns, and many studies focus on prefetching such data with additional space.

However, existing cache replacement policies are not adaptive to the \textit{access patterns} and \textit{cache size}, leading to a sub-optimal situation.
%
We evaluate \XXX{} policies with different cache sizes and workloads, choose the highest hit ratio as the optimal ratio, and compare the hit ratio of the policies with the optimal ratio.
%
As shown in \TODO{figure~\ref{}}, when the cache size is \XXX{} of the working set size(WSS), \XXX{} achieve \XXX{} performance on average.
%
\XXX{} consists of \XXX{} workloads, so there are \XXX{} workloads suffer from a \XXX{} performance drop.
%
In \TODO{figure~\ref{}}, when we increase the cache size, the hit ratio even drops.

\begin{figure}
\centering
\input{data/cachesize}
\end{figure}


This performance degradation is caused by the limitations of access pattern classification and cache design.
%
The access pattern classification does not take cache size into account, causing the same access pattern to be classified into different categories under different cache sizes~(\autoref{ss:access-patterns}).
%
Cache replacement policies, when performing adaptive adjustments, also fail to consider global access information comprehensively~(\autoref{ss:adaption-criteria}). 
%
This leads to design flaws in the adjustment mechanism, resulting in extremely poor performance in certain scenarios.

This paper explores the relationship between access patterns and cache size and proposes \sys, a phase-aware and cache-size-aware replacement policy that can adapt to different access patterns and cache sizes.
%
We design \sys based on three \textit{insights}:
%
First, splitting the access trace into phases according to the cache size is a method for cache-size-aware.
%
Second, to classify the object type dynamically, we consider the access frequency distribution in different phases.
%
Third, the policy should consider the overall cache access behavior to adapt to mixed access patterns.

\TODO{\sys}

\TODO{performance}

In summary, we make the following contributions:
\squishlist
\item{\textbf{todo.}}
\squishend
%