\section{Introduction}
\label{s:intro}
Software caches, e.g., Memcached~\cite{}, Redis~\cite{}, and Linux page cache~\cite{}, are widely adapted in various fields to improve the service performance.
%
In production environments, software cache systems have a fixed memory space and a specific cache replacement policy for all data sets.
%
In the system, there are various workloads with different access patterns, and some of these patterns change over time. 
%
Many works analyze the characteristics of the traces and design the best policy for the system to improve performance.

Cache replacement policies are designed to handle different access patterns adaptively.
%
Prior work~\cite{} classifies the access patterns into four basic types and designs the policy for the mixture of the basic patterns.
%
To design an adaptive policy, they split the cache into different functional parts to filter the interference pattern from the major pattern and duel for two major patterns.
%
Furthermore, some workloads also exhibit regular access patterns, and many studies focus on prefetching such data with additional space.

However, existing cache replacement policies are not adaptive to the \textit{access patterns} and \textit{cache size}, leading to a sub-optimal situation.
%
We evaluate \XXX{13} policies with different cache sizes and workloads, choose the highest hit ratio as the optimal ratio, and compare the hit ratio of the policies with the optimal ratio.
%
As shown in figure~\ref{fig:cloudphysics}(a),  as the cache size increases, the performance of S3FIFO gradually decreases from 95.7\% to 89.6\%, while the performance of LIRS steadily increases from 89.6\% to 96\%.
%
However, as shown in figure~\ref{fig:cloudphysics}(b), the worst-performing 5\% of workloads for both S3FIFO and LIRS only achieve 69.5\% performance, when the cache size is 3\% of the working set size(WSS).
%
While S3FIFO and LIRS perform better overall, they fail to adapt to certain access patterns; in contrast, ARC shows lower overall performance but provides a more balanced effectiveness across different workloads.

\begin{figure}[t]
\centering
\input{data/cachesize}
\caption{Replacement policies' performance in Cloudphysics.}
\label{fig:cloudphysics}
\end{figure}

\begin{figure}[t]
    \centering
    \input{data/mrcdemo}
    \caption{Miss ratio curves and relative performance in systor.}
    \label{fig:mrcdemo}
\end{figure}


This performance degradation is caused by the limitations of access pattern classification and cache design.
%
The access pattern classification does not take cache size into account, causing the same access pattern to be classified into different categories under different cache sizes~(\autoref{ss:access-patterns}).
%
Cache replacement policies, when performing adaptive adjustments, also fail to consider global access information comprehensively~(\autoref{ss:adaption-criteria}). 
%
This leads to design flaws in the adjustment mechanism, resulting in extremely poor performance in certain scenarios.
%
In some policies, as the cache size increases, the hit ratio actually worsens\TODO{figure}.

This paper explores the relationship between access patterns and cache size and proposes \sys, a phase-aware and cache-size-aware replacement policy that can adapt to different access patterns and cache sizes.
%
We design \sys based on three \textit{insights}:
%
First, splitting the access trace into phases according to the cache size is a method for cache-size-aware.
%
Second, to classify the object type dynamically, we consider the access frequency distribution in different phases.
%
Third, the policy should consider the overall cache access behavior to adapt to mixed access patterns.

\TODO{\sys}

\TODO{performance}

In summary, we make the following contributions:
\squishlist
\item{\textbf{todo.}}
\squishend
%