\section{Background}
\label{s:background}
Software caches are wildly implemented in various systems to handle diverse data access patterns.
%
In the KV-store, CDN, and memory devices, the cache stores Key-Value pairs, web objects, and data blocks, respectively, and we use the term \textit{object} to represent the data unit stored in the cache.
%
We have many tools to analyze the trace, determine the access pattern, and design the best cache replacement policy for a specific cache size. 
%
This section introduces the basic access patterns~(\S\ref{ss:access-patterns}) and how prior work designs the cache replacement policy for them~(\autoref{ss:static-criteria}).

\subsection{Access patterns}
\label{ss:access-patterns}
Prior work classifies the access patterns into four basic types and a mixture of the basic types.

\textbf{LRU-friendly} \dots

\textbf{LFU-friendly} \dots

\textbf{Scan} \dots

\textbf{Period repeated access} \dots

\textbf{Mixture of the basic types} \dots

\subsection{Static criteria for mixed access patterns}
\label{ss:static-criteria}
Prior work splits the cache into different parts to handle access patterns or filter interference between them.
%
They change the part size according to the hit information and tag the object type with static criteria.

LRU-friendly and scan-resistance. \dots

LFU-friendly and period repeated access-resistance. \dots

Dueling for LRU-friendly and LFU-friendly. \dots

One-hit-wonder, many objects accessed only once. \dots

