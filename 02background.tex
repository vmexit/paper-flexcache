\section{Background}
\label{s:background}
Software caches are wildly implemented in various fields to handle different access patterns, but their adaptability to mixed access patterns are limited.
%
In the KV-store, CDN, and memory devices, the cache stores Key-Value pairs, web objects, and data blocks, respectively, and we use the term \textit{object} to represent the data unit stored in the cache.
%
We have many tools to analyze the trace, determine the access pattern, and design the best cache replacement policy for a specific cache size. 
%
This section introduces the basic access patterns~(\S\ref{ss:access-patterns}) and how prior work designs the cache replacement policy for them~(\autoref{ss:adaption-criteria}).

\subsection{Access patterns}
\label{ss:access-patterns}
Prior work classifies the access patterns into four basic types and a mixture of the basic types. 
%
This classification method is very intuitive, but we can determine the trace type only after the analysis (e.g., implementing the replacement policy).

\textbf{LRU-friendly} is a set of access that the \cc{least recently used} (LRU) replacement policy is the best choice.

\textbf{LFU-friendly} is a set of access that the \cc{least frequently used} (LFU) replacement policy is the best choice.

\textbf{Scan} is a set of access that all objects are accessed only once, which means no replacement policy can deal with it.

\textbf{Period repeated access} is a set of repeated accesses, and a more restrictive definition is that the objects are accessed with equal probability~\cite{}(cacheus), which means the hit ratio is related to the cache size. If the cache size is larger than the repeated objects, the hit ratio is 100\%, and if the cache size is smaller than the repeated objects, the hit ratio is a fixed number.

%\textbf{Mixture of the basic types} \dots
%This classification method is neither object-based nor access sequence-based, but only for the trace type.
This classification method cannot tell us the object type, only the trace type, and even if the cache size changes, the trace type will change.
%so defining a mixture of the basic types is confusing. 
%
If an object is accessed many times in a short sequence, without access for a long sequence, and then reaccessed, we cannot determine whether it is LRU-friendly or LFU-friendly. 
%
If the footprint of the long access sequence is smaller than the cache size, is the object both LRU-friendly and LFU-friendly? 
%
This classification method is retrospective — the first access to each object is treated as part of a scan, and only after the second access can the object be classified as another type.
%
For trace $\text{AAAABBCCCDDCCA}$, if the cache size is 2, the base policy is LRU, so it is LRU-friendly, but if the cache size is 3, the base policy is LFU, so it is LFU-friendly.
%

\textbf{Summary: We nned a method to classify the object type with cache size awareness.}


\subsection{Adaption for mixed access patterns}
\label{ss:adaption-criteria}
Prior work splits the cache into different parts to handle access patterns or filter interference between them.
%
They change the part size according to the hit information and tag the object type with static criteria.

LRU-friendly and scan-resistance. \dots

LFU-friendly and period repeated access-resistance. \dots

Dueling for LRU-friendly and LFU-friendly. \dots

One-hit-wonder, many objects accessed only once. \dots

