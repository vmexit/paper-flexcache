\section{The \sys Design}
\label{s:design}
Inspired by \autoref{s:phase-hotness}, we presents \sys, a phase- and hotness-based adaptive cache eviction algorithm that covers access patterns space reliably.
%
This section presents \sys's design goals and challenges~(\autoref{ss:design-goals-and-challenges}), an overview of \sys~(\autoref{ss:overview}), and concludes with a discussion~(\autoref{ss:discussion-and-limitations}).

\subsection{Design Goals and Challenges}
\label{ss:design-goals-and-challenges}
We design \sys to meet the following goals and resolve the corresponding challenges.

\textbf{Online patterns identification.}
%
To receive a timely and accurate access pattern, \sys works in an online optimization with FIFO queues and a sketch for phase classification~(\autoref{ss:adaptive-phases}), hotness distribution for hotness classification~(\autoref{ss:hotness-during-phases}), and adjusts patterns with cache size.

\textbf{Algorithm for patterns.}
%
To reduce the analysis overhead, \sys integrates with the online pattern identification.
%
\sys also separates function parts to handle different characteristics of patterns and leverages a suspicious area to confirm the changes of data types for reliability.

\textbf{Efficiency and scalability.}
%
An excellent eviction algorithm should be efficient and scalable.
%
\sys has a high hit ratio to reduce the bandwidth consumption to the backend storage.
%
\sys updates metadata and types when needed to reduce operations for high throughput, and it leverages FIFO queues and a sketch for lock-free implementations to improve scalability.


\subsection{\sys Overview}
\label{ss:overview}

\textbf{\sys components.} Figure~\ref{fig:overview} shows the components and wrokflow of \sys. 
%
The filter~(\ABB{F}), main~(\ABB{M}), and suspicious~(\ABB{S}) queue store metadata and values, while the ghost queue~(\ABB{G}), and the sketch~(\ABB{K}) store metadata.
%
To keep relative access order for phase awareness, all the queues (\ABB{F}, \ABB{M}, \ABB{S}, \ABB{G}) are FIFO-based with a fixed size.
%
Extra metadata includes hotness counter for hotness awareness and suspected field for type changing.
%
\sys records the hotness distribution in an array for all the data in the cache, including metadata, and adjusts the hotness threshold field.
%
\ABB{F} filters out \textit{infrequent-ephemeral} data quickly, and \ABB{M} holds most \textit{frequent-persistent} data.
%
\ABB{G} suggests \textit{frequent-ephemeral} data, and \ABB{K} helps to find \textit{infrequent-persistent} data.
%
During the above process, \sys makes educated guesses about some \textit{frequent-ephemeral} data as \textit{frequent-ephemeral} or \textit{infrequent-persistent} data, moving them to \ABB{S} for observation.
%
\ABB{S} confirms suspected data and checks data types from \ABB{M} at the end of a phase.

\textbf{Workflow of \sys.}
%
\BC{1} For phase analysis, \ABB{K} records all the data movement out of \ABB{F} and checked data by \ABB{S} with a forget function.
%
\BC{2} For hotness analysis, \sys updates the hotness distribution and adjust the hotness threshold field whenever a cache hit or eviction occurs.
%
We will discuss the phase and hotness in \sys later in details.

For the data in \ABB{F} and \ABB{G}, \BC{3} All new data not in the cache are inserted into \ABB{F}.
%
\BC{4} \sys does not move data in \ABB{F} directly when its type changes.
%
Instead, \sys moves \textit{frequent-ephemeral} data to \ABB{M} and \textit{infrequent-ephemeral} data to \ABB{G} when \ABB{F} exceeds its size limit.
%
\BC{5} If hit occurs in \ABB{G}, \sys promotes \textit{frequent-ephemeral} data to \ABB{M} and evicts data when \ABB{G} exceeds its size limit.
%
\BC{6} \sys makes educated guesses about \textit{infrequent-ephemeral} data: when a hit occurs in \ABB{G}, it assumes the data belongs to the \textit{frequent-ephemeral};
%
when data is evicted from \ABB{F}, it assumes to be \textit{infrequent-persistent} with the help of \ABB{K}.
%
\sys places the suspected data into \ABB{S} for verification, while retaining its metadata in \ABB{G} for phase analysis.
%

For the data in \ABB{M} and \ABB{S}, \BC{7} when \ABB{M} exceeds its size limit, \sys moves data in \ABB{M} to \ABB{S} for type verification.
%
\BC{8} In \ABB{S}, \sys checks whether data from \ABB{M} is accessed again in the last phase.
%
If hit occurs, \sys confirms it as \textit{frequent-persistent} data and promotes it to \ABB{M}; otherwise, reduces its hotness counter, evicting it if the counter reaches zero.
%
\BC{9} For suspected data, \sys checks whether it is accessed again in \ABB{S} as an observation window, and promotes it to \ABB{M} if hit occurs; otherwise, it is evicted from \ABB{S}.

\textbf{Phases Discussion.}
%
\sys integrates phase analysis with cache size and maintains data in the cache for at least one phase.
%
\ABB{M} occupies most of the cache space, \ABB{F} and \ABB{S} are small part of the cache, and \ABB{G} has the same number of metadata as the total size of other queues.
%
All the queues in \sys are FIFO-based which maintains the relative access order as discussed in \autoref{ss:adaptive-phases}.
%
Therefore, data in \ABB{M} or \ABB{G} experiences a phase size with a footprint approaching cache size, and we discuss the workflow between the queues for each classification.

As shown in figure~\ref{fig:phaseflow}, there are two phase flows for fixed classification data: \textit{data-based} and \textit{metadata-based}.
%
(1) The \textit{infrequent-ephemeral} data experiences a \textit{metadata-based} phase in \ABB{F} and \ABB{G}, and if there is no type change, it is evicted from \ABB{G} at the end of the phase~(\BC{3},\BC{4},\BC{5}).
%
(2) The other types of data experience a \textit{data-based} phase in \ABB{M} and \ABB{S}, and \sys examins its type in \ABB{S} at the end of the phase~(\BC{7},\BC{8}).

When type transitions or guess happen, there are three phase flows.
%
(1) For certainty transitions, data moves to the \ABB{M} from \ABB{F} or \ABB{G} and enjoys a full phase as \textit{frequent} data~(\BC{4},\BC{5}).
%
(2) For suspected transitions, if it is a right guess, data moves to \ABB{M} from \ABB{S} and enjoys a full phase as \textit{frequent} data; otherwise, it is evicted from \ABB{S}~(\BC{6},\BC{9}).
%
(3) The false guess has a duplicate metadata in \ABB{G}~(\BC{6}), at least experiencing a full phase as \textit{infrequent} data in \ABB{G}, which has multiple chances to be guessed or promoted.

\ABB{K} collects phase information beyond a phase flow in the cache.
%
There are two major phase flows in \sys for \ABB{M} and \ABB{G}, and data will experience at least one phase in them.
%
Therefore, a basic version of \ABB{K} only needs to record data movement out of \ABB{M} and \ABB{G}.
%
While \ABB{K} targets for \textit{infrequent-persistent} data, to reduce the influence of other types, \ABB{K} delays recording data out of \ABB{M} and records data checked by \ABB{S}.
%
To reduce the influence of suspected data, \ABB{K} ignores it and records data into \ABB{G}.
%
Finally, \ABB{K} leverages a forget function to forget old records beyond many phases.


\textbf{Hotness Discussion.}
%
\sys also integrates hotness analysis with cache size and adjust the hotness threshold field adaptively.
%
The hotness distribution records the hotness counter of all data in the cache, including metadata, in an array.
%
As discussed in \autoref{ss:hotness-during-phases}, hotness threshold is a portion of the hottest data in the hotness distribution.
%
In \sys, the target portion is the proportion of number of data stored in \ABB{M}, \ABB{S}, and \ABB{F} to the total amount of metadata data in cache.

\sys have an online hotness analysis.
%
In \sys, the hotness distribution represents the distribution of all hotness counter values, and \sys dynamically adjusts the hotness threshold based on this distribution and the target portion.
%
Whenever a cache hit occurs, \sys increases the hotness counter of the corresponding data, which has a maximum value.
%
Then, \sys updates the data type based on the hotness threshold if hit occurs in \ABB{F} or \ABB{G}. 
%
\sys only decreases the hotness counter when the object is evicted from \ABB{G} or aged in \ABB{S}.

\textbf{Type identification.}
%
To reduce the operation overhead for efficiency, \sys only updates data type when needed, eventhough online algorithm can identify data type timely.
%


\BC{1} as \textit{infrequent-ephemeral} type

%\textbf{Adaption and parameters in \sys.}
%


\textbf{Implementation}
efficiency and scalability

If the hotness counter of an object in \ABB{F} or \ABB{G} exceeds the threshold, \sys promotes it to \ABB{M} as \textit{frequent-ephemeral} type, which is likely to be \textit{frequent-persistent} type.
%
\BC{4} If hit occurs in \ABB{G}, \sys suspects that the corresponding data might be \textit{frequent-ephemeral} type.
%
Based on the \ABB{K} information, \sys suspects that some data in \ABB{F} may be \textit{infrequent-persistent} type.
%
Then \sys moves suspected data with a suspected field to \ABB{S} for further examination and keeps metadata in \ABB{G} for phase analysis.
%
\BC{5} 
%
For queues that exceed their size limit, \sys moves \textit{infrequent-ephemeral} data from \ABB{F} to \ABB{G}, places all data from \ABB{M} into \ABB{S} for verification, and evicts the excess \textit{infrequent-ephemeral} data from \ABB{G}.
%
\BC{6} \ABB{S} checks the suspected data from \ABB{F} and \ABB{G} and the last phase data from \ABB{M}.
%
If cache hits suspected data, \sys confirms suspected \textit{frequent-ephemeral} data from \ABB{G} and \textit{infrequent-persistent} data from \ABB{K} and promotes them to \ABB{M}.
%

For simplicity, the suspicious cache is a small part of the main cache, so we move the objects in the main cache to the suspicious cache and use the aging function to judge whether objects in the suspicious cache are evicted or promoted to the main cache when eviction occurs.
%





\subsection{Discussion and Limitations}
\label{ss:discussion-and-limitations}
%overhead
%discussion about prefetching
%limitation in different case
Adaption //hotness and phase changes
parameters/ sketch, cache size
overhead
prefetching
limitations


\begin{figure}[t]
    \centering
    \includegraphics[width=\columnwidth]{fig/overview.drawio.pdf}
    \caption{Overview for \sys. Cache space consists of five parts: filter cache, main cache, suspicious cache, ghost cache, and CBF. The workflow has three major parts, (1)cache promotion and eviction in four sub-cache. (2)IA-MP objects in the filter cache duels with the tail object in the suspicious cache with CBF. (3)Leverage the metadata in the ghost cache to find FA-FP objects and prefetch sequence accessed objects.\TODO{update}}
    \label{fig:overview}
\end{figure}

\begin{figure}[t]
    \centering
    \includegraphics[width=\columnwidth]{fig/phaseflow.drawio.pdf}
    \caption{\sys consists of two phase flows.(1)Frequently accessed objects' movement from the main cache to the suspicious cache and reinsertion to the main cache. (2)Infrequently accessed objects' movement from the filter cache to the ghost cache. Objects' type changes with suspection in the filter and ghost cache.\TODO{update}}
    \label{fig:phaseflow}
\end{figure}
