\section{The \sys Design}
\label{s:design}
Inspired by \autoref{s:phase-hotness}, we presents \sys, a phase- and hotness-based adaptive cache eviction algorithm that covers access patterns space reliably.
%
This section presents \sys's design goals and challenges~(\autoref{ss:design-goals-and-challenges}), an overview of \sys~(\autoref{ss:overview}), and concludes with a discussion~(\autoref{ss:discussion-and-limitations}).

\subsection{Design Goals and Challenges}
\label{ss:design-goals-and-challenges}
We design \sys to meet the following goals and resolve the corresponding challenges.

\textbf{Online patterns identification.}
%
To receive a timely and accurate access pattern, \sys works in an online optimization with FIFO queues and a sketch for phase classification~(\autoref{ss:adaptive-phases}) and access frequency distribution for hotness classification~(\autoref{ss:hotness-during-phases}).

\textbf{Algorithm for patterns.}
%
To reduce the analysis overhead, \sys integrates with the online pattern identification.
%
\sys also separates function parts to handle different characteristics of patterns and leverages a suspicious area to confirm the changes of data types for reliability.

\textbf{Efficiency and scalability.}
%
An excellent eviction algorithm should be efficient and scalable.
%
\sys has a high hit ratio to reduce the bandwidth consumption to the backend storage.
%
\sys updates metadata and types when needed to reduce operations for high throughput, and it leverages FIFO queues and a sketch for lock-free implementations to improve scalability.


\subsection{\sys Overview}
\label{ss:overview}

\textbf{\sys components.} Figure~\ref{fig:overview} shows the components and wrokflow of \sys. 
%
The filter~(\ABB{F}), main~(\ABB{M}), and suspicious~(\ABB{S}) queue store metadata and values, while the ghost queue~(\ABB{G}), and the sketch~(\ABB{K}) store metadata.
%
Extra metadata includes hotness counter only, and \sys records the access frequency distribution in an array and adjusts the hotness threshold field.
%
\ABB{F} filters out \textit{infrequent-ephemeral} data quickly, and \ABB{M} holds most \textit{frequent-persistent} data.
%
\ABB{G} suggests \textit{frequent-ephemeral} data, and \ABB{K} helps to find \textit{infrequent-persistent} data.
%
\ABB{S} checks and confirms the type changes of data in the cache at the end of a phase.

\textbf{Workflow of \sys.}

\textbf{Phases in \sys.}

\textbf{Adaption and parameters in \sys.}

\textbf{Implementation}


\subsection{Discussion and Limitations}
\label{ss:discussion-and-limitations}
%overhead
%discussion about prefetching
%limitation in different case


This section describes the design of \sys, a phase-aware and cache-size-aware replacement policy to be the optimal replacement policy in all cache sizes.
%
We will discuss the overview of \sys in ~(\autoref{ss:overview}), and how it handles the FA-MP and IF-FP objects in ~(\autoref{ss:filter-hold}).
%
With limited space, we leverage the ghost cache to store the metadata of the evicted objects to find FA-FP objects and prefetch sequence accessed objects in ~(\autoref{ss:ghost}).
%
We also introduce an extra suspicious area to record IA-MP objects in ~(\autoref{ss:dueling}).
%
Finally, we discuss how to integrate \sys with phase-aware, dynamically adjust the criteria, and prefetch objects in ~(\autoref{ss:integrate}).

\begin{figure}[t]
    \centering
    \includegraphics[width=\columnwidth]{fig/overview.drawio.pdf}
    \caption{Overview for \sys. Cache space consists of five parts: filter cache, main cache, suspicious cache, ghost cache, and CBF. The workflow has three major parts, (1)cache promotion and eviction in four sub-cache. (2)IA-MP objects in the filter cache duels with the tail object in the suspicious cache with CBF. (3)Leverage the metadata in the ghost cache to find FA-FP objects and prefetch sequence accessed objects.}
    \label{fig:overview}
\end{figure}

\begin{figure}[t]
    \centering
    \includegraphics[width=\columnwidth]{fig/phaseflow.drawio.pdf}
    \caption{\sys consists of two phaseflows.(1)Frequently accessed objects' movement from the main cache to the suspicious cache and reinsertion to the main cache. (2)Infrequently accessed objects' movement from the filter cache to the ghost cache. Objects' type changes with suspection in the filter and ghost cache.}
    \label{fig:phaseflow}
\end{figure}

\subsection{Overview}
\label{ss:overview}
\sys separates the cache into five parts to handle different types of objects as in figure~\ref{fig:overview}.
%

\textbf{A filter cache} to record the behavior of new objects and evict IA-FP objects quickly.

\textbf{A main cache} to hold the most FA-MP objects and true FA-FP and IA-MP objects.

\textbf{A suspicious cache} to keep the suspicious FA-FP and IA-MP objects for a while.

\textbf{A ghost cache} to store the metadata of the evicted objects, help to find FA-FP objects, and prefetch for sequence accessed objects.

\textbf{A counting bloom filter(CBF)} to discover the IA-MP objects, which the ghost cache cannot cover.

%We mark the object types as frequently accessed in all phases, continuously accessed in some phases, sparsely accessed in some phases, and 
The workflow of \sys is as follows:
%
All the sub-caches in the \sys are FIFO-based to emulate the sliding window discussed in~(\autoref{ss:split-trace}) with a fixed capability. 
%
%
\BC{1} All new objects and prefetched objects are inserted into the filter cache.
%
\BC{2} Whenever a cache hit occurs, we record the \textbf{access frequency distribution} of all the objects in the cache and adjust the criteria for FA objects.
%
We only decrease the access frequency when the object is evicted from the ghost or ages in the suspicious cache.
%
\BC{3} If the access frequency of an object in the filter or ghost cache exceeds the threshold, it will be promoted to the main cache
%
\BC{4} Objects that hit the ghost cache and objects in the filter suspicious by the CBF will be placed into the suspicious cache.
%
\BC{5} When eviction occurs, objects in the filter are moved to the ghost cache, while objects in the ghost cache are discarded.
%
\BC{6} For simplicity, the suspicious cache is a small part of the main cache, so we move the objects in the main cache to the suspicious cache and use the aging function to judge whether objects in the suspicious cache are evicted or promoted to the main cache when eviction occurs.
%
\BC{7} CBF records all objects' movement in the filter, ghost, and suspicious cache.
%

\subsection{Fliter and hold for important (FA-MP) objects}
\label{ss:filter-hold}
\textbf{Group 1: the filter, main and suspicious cache.}
%
\sys uses the filter, main, and suspicious cache to hold FA-MP and filter out IA-FP objects. 
%
\sys puts the new objects into the filter cache and records the access frequency of the objects in the filter cache.
%
If it is frequently accessed, move it to the main cache; otherwise, to the ghost cache with metadata and access frequency.
%
As time passes, more objects fill the main and suspicious cache.
%
\sys ages them in the suspicious cache, checks if they are still frequently accessed (as FA-MP), and then moves them into the main cache or discards them.

\textbf{Phase in the main and suspicious cache.}
%
As shown in figure~\ref{fig:phaseflow}, phase changing happens in the suspicious cache with the aging function.
%
As the basic design in~(\autoref{ss:adaption-criteria}), to filter out the IA-FP objects, the filter cache occupies a small part of the space, while the main cache occupies most of the space. 
%
When an object moves from the head to the tail of the main cache, it has typically persisted for nearly an entire phase. 
%
At that point, aging the object helps determine whether it belongs to a multi-phase access.
%
FA-MP objects contributes the most to the hit ratio~(\autoref{ss:dynamic-adjust}), and suspicious FA-FP and IA-MP objects are small part of the suspicious cache.
%
To make efficient use of space, \sys treats the main cache and the suspicious cache together as the phase duration for frequently accessed data. 
%
The object is not aged when moving from the main cache to the suspicious cache; it is only aged when it reaches the tail of the suspicious cache. 
%
\sys uses a dueling mechanism to reduce interference from suspicious data on primary data while allowing the primary data to stay in the cache for a longer duration~(\autoref{ss:dueling}).

\subsection{Ghost for misjudged (FA-FP) objects and prefetch}
\label{ss:ghost}
\textbf{Group 2: the filter, suspicious and ghost cache.}
%
The filter, suspicious, and ghost cache is a group to find FA-FP objects.
%
\sys leverages the ghost cache to store the metadata of the evicted objects.
%
Some FA-FP objects will escape from the small filter cache to the ghost cache.
%
When a request hits the ghost cache, \sys moves the object to the suspicious cache to observe whether it is misjudged. 
%
When it is accessed again in the suspicious cache, \sys confirms that it is an FA-FP object and will be promoted to the main cache.
%
In another case, when its access frequency exceeds the threshold in the ghost cache, \sys will promote it to the main cache.
%
When sequenced object accesses occur, \sys prefetch the next object from the ghost cache to the filter cache, which separates suspicious objects with hits from the suspicious objects without access.
%
%We duplicate the metadata to avoid prefetching failures from disrupting the relative order of phases~(\autoref{ss:duplicate-metadata}).

\textbf{Phase in the ghost cache.}
The filter and ghost cache objects consist of infrequently accessed (IA) objects within a phase.
%
In prior work~\cite{}, the ghost cache contains the same amount of metadata as the main cache to avoid misjudgments~(\autoref{ss:adaption-criteria}), and they assumed that all request hit the ghost cache are FA-FP objects.
%
To identify whether any frequently accessed object (FA) has been mistakenly overlooked, \sys considers, from a global perspective~(\autoref{ss:integrate}).
%
The method requires maintaining the relative order of their first access and retain these objects for one phase in figure~\ref{fig:phaseflow}. 
%
However, when a request hits the ghost cache, objects movements as prior work disrupt this order, so \sys duplicate the metadata to avoid interference caused by misjudgments~(\autoref{ss:integrate}). 
%
Maintaining this relative access order also facilitates effective prefetching during sequential accesses.
%
If they are FA-FP objects, management responsibility is transferred to the main cache, and only then does the ghost cache evict thes objects.


\subsection{Dueling for suspicious (IA-MP) objects}
\label{ss:dueling}
\textbf{Group 3: the filter, suspicious cache and CBF.}
%
In addition to FA-MP and FA-FP objects, \sys uses a Counting Bloom Filter (CBF) to handle IA-MP objects.
%
These object accesses have intervals close to or greater than one phase, and storing them in the ghost cache would require significant space. 
%
To reduce space overhead, \sys uses a CBF to record access patterns over multiple phases. 
%
When an object is about to leave the filter cache, \sys compares its access record in the CBF with the access record of the tail object in the suspicious cache.
%, which is the object that will be evicted from the main cache.
If it wins the duel, \sys puts the object into the suspicious cache and evict the tail object from the suspicious cache.

The information recorded in the CBF is subject to some inaccuracies, so we apply three optimizations to limit its impact. 
%
First, \sys periodically ages the data in the CBF to minimize the influence of objects accessed many phases ago. 
%
Second, to avoid redundant recording of frequently accessed objects, \sys only records objects once for each phase in the CBF, as in figure~\ref{fig:phaseflow}.
%
To make it timely, \sys records the objects when they leave the filter cache or suspicious cache.
%
Finally, \sys turns off this feature if the record is too large or the object in the suspicious cache is to be promoted to the main cache.

\subsection{Integrate \sys with phase aware}
\label{ss:integrate}
\sys adopts a FIFO-based eviction approach to achieve phase awareness because it preserves the relative order of first accesses.
%\sys adopts a FIFO-based eviction approach to achieve phase awareness.% and duplicates metadata for suspicious objects to preserve access order.
%
As mentioned in figure~\ref{fig:phaseflow}, \sys involves two types of phase-aware objects handling: (1)frequently accessed objects undergo phase aging in the main cache and suspicious cache~(\autoref{ss:filter-hold}). (2)the ghost cache handles infrequently accessed objects with phase awareness~(\autoref{ss:ghost}).
%
Each metadata includes an access counter that increments on each access and decrements during aging, so the object will be evicted if it hasn't been accessed over multiple phases.
%
The access frequency distribution of all objects in the cache represents the access pattern of the current phase, and \sys uses this information to adjust the promotion threshold dynamically.
%This allows \sys to know the access frequency distribution of all objects in the cache. 
%

\sys distinguishes between frequently and infrequently accessed objects based on overall access behavior. 
%
We assume that high-frequency data resides in the main and suspicious cache, and we use their frequency as a threshold. 
%
Only objects in the filter cache with an access count exceeding this threshold will be promoted to the main cache.
%
If the aforementioned situations occur~(\autoref{ss:adaption-criteria}), we dynamically raise the promotion threshold to prevent cache pollution.
%
This approach avoids data calcification, where data in the main cache is never evicted and new data cannot be retained.  
%
First, the filter cache still contains a portion of frequently accessed objects, which keeps the promotion threshold relatively lenient. 
% 
Second, when an object hits ghost cache data, it is placed in the suspicious cache, which occupies part of the space with suspicious objects. 
%
Third, the CBF records the phase information, its dueling between the filter and suspicious cache will disturb data calcification.

suspicious objects may disrupt the relative order, so \sys duplicates the metadata as a placeholder in figure~\ref{fig:phaseflow}.
%
There are three sources of suspicious objects in the policy: objects generated by prefetching~(\autoref{ss:ghost}), requests hitting objects in the ghost cache~(\autoref{ss:ghost}), and potential IA-MP objects in the filter cache~(\autoref{ss:dueling}). 
%
If we wrongly promote the suspicious objects, and then evict them back to the ghost cache, they persists for a longger phase.
%
Therefore, \sys duplicates the metadata for suspicious objects to avoid disrupting the relative order of phases.
%
If they are truely FA-MP objects, \sys discards the metadata in the ghost cache and promotes them to the main cache.

\sys decouples the suspicious objects \textbf{with hits} from the suspicious objects \textbf{without access}.
%
The suspicious objects with hits are promoted to the suspicious cache, and it will become a FA-MP object if it is accessed again.
%
While \sys prefetches the suspicious objects if prior requests access in sequence.
% 
Therefore, the prefetch object is placed in the filter cache to avoiding pollution.
%
We assume that if a prefetch is effective, it will likely be effective multiple times.
%
\sys increase its access count after multiple hits, or it will be evicted from the filter cache to save the cache space.

